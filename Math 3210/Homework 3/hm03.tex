\documentclass[12pt, letterpaper]{article}
\usepackage[utf8]{inputenc}
\usepackage{mathtools}
\usepackage{amsfonts}
\usepackage{amsmath}
\usepackage{amsthm}
\usepackage{amssymb}
\usepackage{mathtools}
\usepackage[scr]{rsfso}

\title{Homework Assignment 3}
\author{Joshua Cragun \thanks{u1025691} \\ Prof. Gordan Savin, MATH 3210}
\date{October 2018}

\begin{document}

\begin{titlepage}
\maketitle
\end{titlepage}

\pagebreak

\section*{Question 1}
Let $x_n$ be a sequence of positive real numbers such that $\lim_n x_n =x>0$. Prove that
\begin{itemize}
\item
$\lim_n x_n^2= x^2$.
\item
$\lim_n \sqrt{x_n}= \sqrt{x}$.
\end{itemize}

\noindent\textbf{Part 1:} Using the identity
$$ x_n^2 - x^2 = (x_n - x)^2 + 2x(x_n - x) $$
Where for any given $\varepsilon > 0$ there exists an integer $N$ such that for any $m \geq N$, $|x_m - x| < \sqrt{\varepsilon}$.
This implies that $|(x_n - x)^2| < \varepsilon$ and therefore $\lim_{n\to\infty} (x_n - x)^2 = 0$. And clearly
$$\lim_{n\to\infty} 2x(x_n - x) = 2x \cdot 0 = 0$$
Thus $\lim_n x_n^2 - x^2 = \lim_n (x_n - x)^2 + 2x(x_n - x) = 0$ and therefore
$$ \lim_n x_n^2 = x^2 $$
\qed
\\
\noindent\textbf{Part 2:}

\noindent By what was given, one need not consider the cases where $x \leq 0$. If $x > 0$ then there exists an $N$ such that if $m \geq N$, $|x_m - x| < \varepsilon\sqrt{x}$. Then, because this is over positive real numbers,
$$ |\sqrt{x_m} - \sqrt{x}| = \frac{|x_m - x|}{|\sqrt{x_m} + \sqrt{x}|} < \frac{|x_m - x|}{\sqrt{x}} < \frac{\varepsilon\sqrt{x}}{\sqrt{x}} = \varepsilon $$
Thus $\lim_n \sqrt{x_n}= \sqrt{x}$.\qed

\pagebreak

\section*{Question 2}
Define a sequence by $s_1=1$ and $s_{n+1} = \sqrt{2+\sqrt{s_n}}$. Prove that $s_n <2$ for all $n$, and that $s_n$ is an increasing sequence. Find the limit.\\

\noindent To begin, one sees immediately that $s_n$ increases between $s_1 = 1$ and $s_2 = \sqrt{3}$. Suppose that $s_n$ is increasing from $n = 1, ..., m$. Then,

\begin{align*}
  s_m = \sqrt{2 + \sqrt{s_{m-1}}} &> s_{m-1}\\
  \sqrt[4]{2 + \sqrt{s_{m-1}}} &> \sqrt{s_{m-1}}\\
  2 + \sqrt[4]{2 + \sqrt{s_{m-1}}} &> 2 + \sqrt{s_{m-1}} \\
  \sqrt{2 + \sqrt[4]{2 + \sqrt{s_{m-1}}}} = s_{m + 1} &> \sqrt{2 + \sqrt{s_{m-1}}} = s_m
\end{align*}

Hence $s_{m + 1} > s_m$. Thus the sequence is increasing for all $n \geq 1$. Next, suppose that some $s_m \geq 2$. Then,

\begin{align*}
  2 &\leq \sqrt{2 + \sqrt{s_{m - 1}}}\\
  4 &\leq 2 + \sqrt{s_{m - 1}}\\
  2 &\leq \sqrt{s_{m - 1}} \\
  4 &\leq s_{m - 1}
\end{align*}

Thus, if $s_m \geq 2$ then $s_{m - 1} \geq 4$. But this contradicts the fact that the sequence is strictly increasing. Thus $s_n <2$ for all $n \geq 1$.
To find the limit of the sequence, let $\lim_n \sqrt{2+\sqrt{s_n}} = L$. Then
$$ L = \sqrt{\lim_n (2 + \sqrt{s_n})} = \sqrt{2 + \lim_n \sqrt{s_n}} = \sqrt{2 + \sqrt{\lim_n s_n}} = \sqrt{2 + \sqrt{L}} $$
Thus the limit will be a solution to the expression above, or when rearranged:
$$ L^4 - 4L^2 - L + 4 = 0 $$
Which is approximately 1.83118, or in exact form:
$$\frac{1}{3}\Big( -1 + \sqrt[3]{\frac{1}{2}(79 - 3\sqrt{249})} + \sqrt[3]{\frac{1}{2}(79 + 3\sqrt{249})} \Big)$$
\pagebreak

\section*{Question 3}
Let $X=\mathbb Z$ and $d(x,x)=0$ or $d(x,y)=\frac{1}{2^n}$, if $x\neq y$, where $2^n$ is the largest power of $2$ dividing $x-y$.
Prove that the following two series are Cauchy. One of them is convergent (find its sum) while the other not (explain).
\begin{itemize}
\item $\sum_{n=0}^{\infty} 2^n$
\item $\sum_{n=0}^{\infty} (-2)^n$
\end{itemize}

\noindent To begin, let $s$ denote the first series, and let $r$ denote the second. Consider two nonequal partial sums $s_n$, and $s_m$. Then,
\begin{align*}
  s_n &= 2^0 + 2^1 + \cdots + 2^n \\
  s_m &= 2^0 + 2^1 + \cdots + 2^m
\end{align*}
Without loss of generality, assume $m > n$. Then,
\begin{align*}
  s_m - s_n &= 2^{n + 1} + \cdots + 2^m \\
  &= 2^{n + 1}(1 + 2^1 + \cdots + 2^{m - n - 1})
\end{align*}
Thus the differences between any two nonequal partial sums of degrees $m$ and $n$ with $m > n$ is $\frac{1}{2^{n+1}}$. If $n = m$ then the distance is zero, by definition.
One can easily see that the same is true for $r$ by simply replacing any $2$ in the steps above with $(-2)$ and then factoring out $2^{n+1}$ at the very end.
\par With this is it very straightforward to show that both series are
Cauchy. Given any $\varepsilon > 0$ one can find a $j$ such that $0 < \frac{1}{2^j} < \varepsilon$ and $\varepsilon \leq \frac{1}{2^{j + 1}}$.
Without loss of generality, take any partial sum of degree $k > j$ of $s$, then
$$d(s_k, s_j) = \frac{1}{2^{j + 1}} < \varepsilon$$
It is very easy to see that the statement above is equally true for $r$. Hence, both series are Cauchy.\qed\\

\noindent Although both are Cauchy, only $s$ is convergent, and it converges to -1. The reason is through considerations of the limit of the similar sequence $\lim_{n\to\infty} 2^n$.
One sees that the distance between 0 and any $2^n$ is $\frac{1}{2^n}$ since 0 is divisible by any integer. Hence,
$$ \lim_{n\to\infty} d(0, 2^n) = \lim_{n\to\infty} \frac{1}{2^n} = 0$$
Thus $\lim_{n\to\infty} 2^n = 0$ and since any partial sum of $s, s_n = 2^{n+1} - 1$,
$$\lim_{n\to\infty} s_n = \lim_{n\to\infty} 2^{n + 1} - 1 = 0 - 1 = -1$$
The reason the other does not converge is slightly more nuanced, because in a sense, the series does converge. Consider the sequence of partial sums of $r$, $\{r_n\}$. We have
$$ r_{n\geq0} = \frac{1}{3}((-1)^n(2^{2n} + 1) $$
So Then
$$ \lim_{n\to\infty} r_n = \lim_{n\to\infty} \frac{1}{3}((-1)^n2^{2n} + 1) $$
Where $(-1)^n2^{2n}$ will either be $-(2^{2n})$ or $2^{2n}$. Hence $d((-1)^n2^{2n}, 0)$ is $\frac{1}{2^{2n}}$, which as $n$ increases converges to 0. Thus,
$$ \lim_{n\to\infty} \frac{1}{3}((-1)^n2^{2n} + 1) = \frac{1}{3}(0 + 1) = \frac{1}{3}$$
So if this metric space was over $\mathbb{Q}$ or above, the series would converge, however since $\frac{1}{3} \notin \mathbb{Z}$, the series can not converge, by definition.\qed
\pagebreak
\section*{Question 4}
Let $c_n$ be a sequence of positive numbers. Prove that
\[
\mathrm{liminf}_n \frac{c_{n+1}}{c_n} \leq \mathrm{liminf}_n \sqrt[n]{c_n}.
\]
\begin{proof}
  Let $\alpha'= \mathrm{liminf}_n \sqrt[n]{c_n}$ and $\alpha = \mathrm{liminf}_n \frac{c_{n+1}}{c_n}$. Also let $B$ be an arbitrary positive number such that $B < \alpha$ and let $z_m = \mathrm{liminf}_{m \geq n} \frac{c_{m+1}}{c_m}$. Then $\exists N$ such that $z_N \geq B$ for all $m \geq N$. In other words we have,
  $$ B \leq \frac{c_{N + 1}}{c_N}, B \leq \frac{c_{N + 2}}{c_{N + 1}}, \cdots, B \leq \frac{c_{m + 1}}{c_m}$$
  Or,
  $$ c_{N+1} \geq Bc_N, c_{N+2} \geq B^2c_N, \cdots, c_m \geq B^{m - N}c_N $$
  So then in another form, $c_m \geq \frac{c_N}{B^N}B^m$. Then we also have that
  $$ \sqrt[m]{c_m} \geq \sqrt[m]{\frac{c_N}{B^N}B^m} $$
  However note that $\frac{c_N}{B^N}$ is constant, so as $m$ increases, the value of the $m$-th root of the quotient approaches 1, so we then have
  $$ \sqrt[m]{c_m} \geq \sqrt[m]{\frac{c_N}{B^N}B^m} \to \sqrt[m]{B^m} = B$$
  Thus for all $B < \alpha$, $\alpha' \geq B$. This implies that $\alpha' \geq \alpha$. In other words,
  \[
  \mathrm{liminf}_n \sqrt[n]{c_n} \geq \mathrm{liminf}_n \frac{c_{n+1}}{c_n}.
  \]
\end{proof}
\pagebreak

\section*{Question 5}
Let $(X,d)$ be a metric space.  Let $x=\{x_n\}$ and $y=\{y_n\}$ be two Cauchy sequences.
Prove that the sequence of distances $d(x_n,y_n)$ is a Cauchy sequence of real numbers.

\begin{proof}
  Since $\{x_n\}$ and $\{y_n\}$ are both Cauchy, it is true that for any $\varepsilon > 0$, $\exists N$ such that $d(x_n, x_m) < \frac{\varepsilon}{2}$ for all $m \geq N$ (and likewise for $\{y_n\}$).
  Then because $X$ is a metric space,
  \begin{align}
    d(x_n, y_n) &\leq d(x_n, x_m) + d(x_m, y_m) + d(y_m, y_n)\\
    d(x_n, y_n) &< \frac{\varepsilon}{2} + d(x_m, y_m) + \frac{\varepsilon}{2}
  \end{align}
  By rearranging $(2)$ we get
  $$ d(x_n, y_n) -  d(x_m, y_m) < \varepsilon $$
  And by multiplying $(1)$ by -1 and rearranging, we find that
  $$ d(x_m, y_m) - d(x_n, y_n) > -\varepsilon $$
  Which implies that
  $$ |d(x_n, y_n) -  d(x_m, y_m)| < \varepsilon $$
  And since this is the standard distance formula for the real numbers, this shows that the sequence of distances $d(x_n,y_n)$ is a Cauchy sequence in $\mathbb{R}$.
\end{proof}
\pagebreak
\section*{Question 6}
Let $(X,d)$ be a metric space.  Two Cauchy sequences $x=\{x_n\}$ and $y=\{y_n\}$ are equivalent if for every $\epsilon >0$, there exists $n$ such that
$d(x_m,y_m) < \epsilon$ for all $m\geq n$. Prove that this is an equivalence relation.

\begin{proof}
  Since $X$ is a metric space, $d(x, x) = 0$ with any $x \in X$. Hence for any sequence $\{x_n\}$
  $$ d(x_m, x_m) = 0 < \varepsilon$$
  Thus $\{x_n\}$ is equivalent to itself, and thus the relation is reflexive. Next, suppose a sequence $x=\{x_n\}$ was equivalent to another sequence $y=\{y_n\}$.
  Then since $X$ is a metric space,
  $$ d(y_m,x_m) = d(x_m, y_m)< \epsilon $$
  Hence $y$ is equivalent to $x$ and is therefore a symmetric relation. Now suppose a sequence $x=\{x_n\}$ was equivalent to another sequence $y=\{y_n\}$ and suppose that
  $y$ was equivalent to another sequence $z=\{z_n\}$. Then there exists an $N_1, N_2$ such that $d(x_m, y_m) < \frac{\varepsilon}{2}$ and $d(y_m, z_m) < \frac{\varepsilon}{2}$ for $m \geq max(N_1, N_2)$. Then since $X$ is a metric space,
  \begin{align*}
    d(x_m, z_m) &\leq d(x_m, y_m) + d(y_m, z_m)\\
    d(x_m, z_m) &< \frac{\varepsilon}{2} + \frac{\varepsilon}{2} = \varepsilon
  \end{align*}
  therefore $x$ and $z$ are also equivalent and the relation is transitive. Since the reltaion is reflexive, symmetric, and transitive, the relation defines an equivalence relation.
\end{proof}
\pagebreak
\section*{Question 7}
Let $Y$ be a non-empty set and $d: Y \times Y\rightarrow [0, \infty)$ a ``distance'' function such that
\begin{itemize}
\item $d(x,x)=0$  for all $x\in Y$.
\item $d(x,y)=d(y,x)$  for all $x,y\in Y$.
\item $d(x,z) \leq d(x,y) + d(y,z)$ for all $x,y,z\in Y$.
\end{itemize}
In words, $d$ is almost a distance function, however, $d(x,y)=0$ is allowed for different $x$ and $y$. We say that $x$ and $y$ are
equivalent if $d(x,y)=0$. (1) Prove that this is an equivalence relation. (2) Prove that, if $x$ is equivalent to $y$ then $d(x,z)=d(y,z)$ for all $z\in Y$.

\begin{proof}
  Reflexivity and symmetry are given. All that is left to show is transitivity. Let $x$ be equivalent to $y$ and $y$ be equivalent to $z$. Then,
  $$ d(x, z) \leq d(x, y) + d(y, z) $$
  And since $x$ and $y$ are equivalent,
  $$ d(x, z) \leq 0 + d(y, z) = d(y, z)$$
  So $d(x, z) \leq d(y, z)$. Now instead consider $d(y, z)$
  \begin{align*}
    d(y, z) &\leq d(y, x) + d(x, z)\\
    &\leq d(x, y) + d(x, z)\\
    &\leq 0 + d(x, z)
  \end{align*}
  So we have $d(x, z) \leq d(y, z)$ and $d(y, z) \leq d(x, z)$. Thus $d(x, z) = d(y, z)$, immediately proving (2). However if $y$ and $z$ are equivalent, then $d(y, z) = 0$. So then in the case above, $d(x, z) = d(y,z) = 0$, meaning that $x$ must be equivalent to $z$, proving transitivity, and therefore that the relation is an equivalence relation.
\end{proof}



\end{document}

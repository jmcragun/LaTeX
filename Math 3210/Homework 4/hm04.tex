\documentclass[12pt, letterpaper]{article}
\usepackage[utf8]{inputenc}
\usepackage{mathtools}
\usepackage{amsfonts}
\usepackage{amsmath}
\usepackage{amsthm}
\usepackage{amssymb}
\usepackage{mathtools}
\usepackage[scr]{rsfso}

\title{Homework Assignment 4}
\author{Joshua Cragun \thanks{u1025691} \\ Prof. Gordan Savin, MATH 3210}
\date{November 2018}

\begin{document}

\begin{titlepage}
\maketitle
\end{titlepage}

\section*{Question 1}
Let $f: \mathbb R \rightarrow \mathbb R$ such that $|f(x)-f(y)| \leq (x-y)^2$. Prove that $f$ is constant.

\begin{proof}
  As given, we have two cases to consider since $f(x)-f(y) \leq (x-y)^2$ and $f(x)-f(y) \geq -(x-y)^2$.\\

  \noindent In the former, we have
  $$ f'(x) = \lim_{c \to x} \frac{f(c) - f(x)}{c - x} \leq \frac{(c-x)^2}{c-x} = c-x = 0$$
  \noindent In the latter we have
  $$ f'(x) = \lim_{c \to x} \frac{f(c) - f(x)}{c - x} \geq \frac{-(c-x)^2}{c-x} = x-c = 0$$
  So then we have that $f'(x) \geq 0$ and $f'(x) \leq 0$. Hence $f'(x) = 0$, showing that $f$ is constant.
\end{proof}

\section*{Question 2}
Let $f$ be a differentiable function defined in a neighborhood of $x$. Assume  that $f''(x)$ exists. Prove that
\[
\lim_{h\rightarrow 0} \frac{f(x+h) + f(x-h) -2 f(x)}{h^2} = f''(x).
\]

\begin{proof}
  To begin, first consider the derivative for some artbitrary differentiable function $g$.
  $$ g'(x) = \lim_{c \to x} \frac{g(c) - g(x)}{c - x} = \lim_{c \to x} \frac{g(x + (c - x)) - g(x)}{c - x} $$
  Hence if one examines the limit intead with respect to $h = (c - x)$ we get
  $$ g'(x) = \lim_{h \to 0} \frac{g(x + h) - g(x)}{h}$$
  Then, using this notation instead we have
  $$ f''(x) = \lim_{h \to 0} \frac{f'(x + h) - f'(x)}{h} $$
  Where
  \begin{align*}
    f'(x + h) &= \lim_{k \to 0} \frac{f(x + h + k) - f(x + h)}{k}\\
    f'(x) &= \lim_{l \to 0} \frac{f(x + l) - f(x)}{l}
  \end{align*}
  So then if we choose $k = l = -h$, we get
  \begin{align*}
  f''(x) &= \lim_{h \to 0} \frac{ \frac{f(x) - f(x + h)}{-h} - \frac{f(x - h) - f(x)}{-h} }{h}\\
  &= \lim_{h \to 0} \frac{ \frac{f(x + h) - f(x)}{h} + \frac{f(x - h) - f(x)}{h} }{h} \\
  &= \lim_{h \to 0} \frac{f(x+h) + f(x-h) -2 f(x)}{h^2}
  \end{align*}
  As was to be shown.
\end{proof}
\section*{Question 3 (Fixed Point Theorem)}
Let $f: \mathbb R \rightarrow \mathbb R$ such that $|f'(x)| \leq C$ for some $0\leq C< 1$ and all $x$.  A number $x$ is a fixed point for $f$ if
$f(x)=x$. Prove that $f$ cannot have two fixed points.
Let $x_1$ be any real number, and define
a sequence by $x_{n+1}=f(x_n)$. Prove that the sequence $\{x_n\}$ is Cauchy.
\section*{Question 4 (Concavity)}
Let $f: (\alpha,\beta)  \rightarrow \mathbb R$ be twice differentiable function such that $f''\geq 0$ on the interval. Let $c\in (\alpha,\beta)$ and let $g(x)$ be the linear
function whose graph is the tangent line of the graph of $f$ at $c$ i.e. $g(x)=f(c) + f'(c)(x-c)$. Prove that $f(x)\geq g(x)$ for $x\in (\alpha,\beta)$.
\section*{Question 5 (Newton Method)}
Let $f: \mathbb R \rightarrow \mathbb R$ be twice differentiable function. Let $[a,b]$ be a closed interval such that $f(a) <0$ and $f(b)>0$, $f'(x) \geq \delta>0$,
and $f''(x)\geq 0$ for $x\in [a,b]$. Prove that there is unique $c\in (a,b)$ such that $f(c)=0$. Define a sequence by $x_1=b$ and
\[
x_{n+1} = x_n -\frac{f(x_n)}{f'(x_{n})}
\]
Prove that the sequence is decreasing and bounded from below by $c$, it has a limit.
Prove that the limit is $c$.  Check that the conditions are
satisfied for $f(x)=x^2-2$ and the interval
$[1,2]$. What is the limit of the sequence $\{x_n\}$? Compute $x_n$  for $n=1,2,3,4$.
\section*{Question 6}
Consider the power series $x-\frac{x^3}{3!} + \frac{x^5}{5!} - \ldots$, i.e. the sequence whose $n$-th term is $(-1)^{n-1}\frac{x^{2n-1}}{(2n-1)!}$. Compute the
radius of convergence of this series. Use the theorem of Taylor to prove that $\sin(x)=x-\frac{x^3}{3!} + \frac{x^5}{5!} - \ldots$ for every $x$.
Use this series to find a rational number that approximates $\sin(1/2)$ with an error less than $1/10^3$.

\end{document}

\documentclass[12pt, letterpaper]{article}
\setlength{\parskip}{\baselineskip}%
\usepackage[utf8]{inputenc}
\usepackage{mathtools}
\usepackage{amsfonts}
\usepackage{amsmath}
\usepackage{amsthm}
\usepackage{amssymb}
\usepackage{mathtools}
\usepackage[scr]{rsfso}

\title{Homework 2}
\author{Joshua Cragun \thanks{Prof. Savin, Math 3210}}
\date{September 2018}

\begin{document}

\begin{titlepage}
\maketitle
\end{titlepage}

\section*{Question 1}
A complex number $z$ is called \textit{algebraic} if there exists integers $a_0, a_1, ..., a_n$,
such that $a_n z^n + \cdots + a_1 z + a_0$. Prove that the algebraic numbers are countable.

\begin{proof}
  To begin, first consider all polynomials in the form
  $$a_n z^n + \cdots + a_1 z + a_0$$
  From the fundamental theorem of algebra, it is known that there are $n + 1$ complex solutions to the given expression (all of whom are therefore algebraic numbers).
  Let $\mathbb{F}$ be the set of all polynomials with integer coefficients and let $F_i$ be a set of all polynomials with integer coefficients of degree $i$. Then clearly,
  $$ \mathbb{F} = \bigcup\limits_{i = 0}^{\infty} F_i$$
  And clearly this is a countable union, since the $i$th term can be mapped to $i$ in the integers. Next, consider the mapping $f: F_i \mapsto \mathbb{Z}^{i+1}$:
  $$ a_i z^i + \cdots + a_1 z + a_0 \mapsto (a_0, ..., a_i)$$
  Evidently this defines an bijection on $F_i$ since if any outputs in $\mathbb{Z}^{i+1}$ were the same it would indicate the exact same polynomial
  and any $(a_0, ..., a_i) \in \mathbb{Z}^{i+1}$ will corresponds unqiuely to a polynomial in $F_i$ since there are no restraints on range of acceptable coefficients
  to any term of a given polynomial in $F_i$ or $\mathbb{F}$. 

\end{proof}

\end{document}

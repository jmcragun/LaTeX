\documentclass[12pt, letterpaper]{article}
\setlength{\parskip}{\baselineskip}%
\usepackage[utf8]{inputenc}
\usepackage{mathtools}
\usepackage{amsfonts}
\usepackage{amsmath}
\usepackage{amsthm}

\title{Homework 1}
\author{Joshua Cragun \thanks{Prof. Savin, Math 3210}}
\date{August 2018}

\begin{document}

\begin{titlepage}
\maketitle
\end{titlepage}

\section{Question 1}
For $i = 0, ..., 10$, construct the greatest rational number in the cut of the
square root of $3$ in the form of a (non reduced) fraction $a_i = \frac{x_i}{2^i}$,
and the least rational number not in the cut in the same form, $b_i = \frac{y_i}{2^i}$.

\noindent\textbf{Solution:}

To be begin, we shall start with $a_0$ and $b_0$. Recall the definition of a cut bounded by $\sqrt{3}$.
$$ (\sqrt{3})^{*} \coloneqq \{r \in \mathbb{Q} \mid r^2 < 3 \} $$
This will make it quite easy to determine which rational numbers are in and not
in $(\sqrt{3})^{*}$. From the definition of $a_0$ and $b_0$, it is clear that they will
be integers (specifcally, the two closest integers above and below $\sqrt{3}$) since the denominator is simply $2^0$, or one.

Let us then describe an algorithm for defining what the next $a_i$ and $b_i$ will be.
We know that there exists an integer $n$ such that $(\frac{n}{m})^2 < 3$ and $(\frac{n + 1}{m})^2 > 3$ where $m \in \mathbb{N}$ as a consequence of the Archimedean Principle.
Because the binary operation of averaging two numbers naturally multiplies the denominator by $2$ and finds the midpoint between
two bounds (which will be tending towards $\sqrt{3}$), it is optimal for producing the following the following $a_i$ and $b_i$ in the series.

A ``brute force" recurrence relation will then be described as follows:

\begin{enumerate}
  \item Define possible greatest/smallest rational number in/out of the cut of a given denominator $m \in \mathbb{N}$ as $c_i = \frac{n}{m}$ as the average of $a_{i-1}$ and $b_{i-1}$ with $c_0 = \frac{0}{1}$
  \item If $n^2 < 3 \cdot m^2$ increment $n$ by one until $n^2 > 3 \cdot m^2$, then $\frac{n}{m} = b_i$ and $\frac{n - 1}{m} = a_i$
  \item Otherwise $n^2 > 3 \cdot m^2$, then decrement $n$ by one until $n^2 < 3 \cdot m^2$, then $\frac{n}{m} = a_i$ and $\frac{n + 1}{m} = b_i$
\end{enumerate}

\noindent Given the exponential nature of the denominator, this could prove a tedious problem by hand, however computers can execute this algorithm very swiftly.
But, for demonstrative purposes, I shall show how this algorithm finds $a_0$ and $b_0$.

Start with $c_0 = \frac{0}{1}$. Then $c_0^2 = 0 < 3$, so increment the numerator by 1 until $c_0^2 > 3$, which happens after two increments ($c_0 = \frac{2}{1}$) as $c_0^2 = 2^2 = 4 > 3$.

\noindent Then we have:
\begin{equation}
\begin{split}
a_0 & = \frac{n - 1}{m} = \frac{2 - 1}{1} = 1 \\
b_0 & = c_0 = 2
\end{split}
\end{equation}

\noindent Continuing this process for $i = 1, 2, ..., 10$ we get:
\begin{enumerate}
  \item $a_1 = \frac{3}{2}, b_1 = \frac{4}{2}$
  \item $a_2 = \frac{6}{4}, b_2 = \frac{7}{4}$
  \item $a_3 = \frac{13}{8}, b_3 = \frac{14}{8}$
  \item $a_4 = \frac{27}{16}, b_4 = \frac{28}{16}$
  \item $a_5 = \frac{55}{32}, b_5 = \frac{56}{32}$
  \item $a_6 = \frac{110}{64}, b_6 = \frac{111}{64}$
  \item $a_7 = \frac{221}{128}, b_7 = \frac{222}{128}$
  \item $a_8 = \frac{443}{256}, b_8 = \frac{444}{256}$
  \item $a_9 = \frac{886}{512}, b_9 = \frac{887}{512}$
  \item $a_{10} = \frac{1773}{1024}, b_{10} = \frac{1774}{1024}$
\end{enumerate}
\pagebreak
\section{Question 2}
Let $\alpha$ and $\beta$ be two cuts of $\mathbb{Q}^+$. Let $\alpha \cdot \beta = \{ rs \mid r \in \alpha, s \in \beta \}$. Prove that $\alpha \cdot \beta$ is a cut of $\mathbb{Q}^+$.

\begin{proof}

\noindent To show that any given set is a cut, one must show that for a given set $S$:
\begin{enumerate}
  \item $\forall x \in S$, if $y < x$, then $y \in S$.
  \item $S$ has no maximal element.
\end{enumerate}

\noindent\textbf{1.} Let $c = a \cdot b \in \alpha \cdot \beta$ where $a \in \alpha$ and $b \in \beta$. Let $d < c$. Thus $d < a \cdot b$. Then $\frac{d}{b} < a$ (the inequality will not flip since this is over $\mathbb{Q}^+$).
Since $\alpha$ defines a cut and $\frac{d}{b} < a$, $\frac{d}{b} \in \alpha$. Then $d = \frac{d}{b} \cdot b$ where $\frac{d}{b} \in \alpha, b \in \beta$, so $d \in \alpha \cdot \beta$.

\noindent\textbf{2.} Let $c = a \cdot b \in \alpha \cdot \beta$. Since $\alpha$ is a cut, $\exists a_2 \in \alpha: a_2 > a$,
 which then means $a_2 \cdot b \in \alpha \cdot \beta$ and $ a_2 \cdot b > a \cdot b = c$

\noindent And since properties 1 and 2 have been shown, $\alpha \cdot \beta$ must be a Dedekind cut.
\end{proof}
\pagebreak
\section{Question 3}



\end{document}
